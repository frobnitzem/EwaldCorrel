\documentclass{amsart}
\usepackage[letterpaper,top=1in,bottom=1in,%
            footskip=0in]{geometry}

\renewcommand{\vec}[1]{\mathbf{#1}}

\title{ Structure Factor Calculations}
\author{ David M. Rogers}

\begin{document}

  The included code uses the discrete Fourier transform (DFT)
to calculate approximate structure factors,
\begin{equation}
S(\vec m) = \sum_{a=1}^N q_a \exp(-2\pi i \vec m\cdot \vec r_a)
  \approx \hat S(\vec n),
\end{equation}
on grid points $\vec n_\alpha = [0, 1, \ldots, \lfloor K_\alpha/2 \rfloor, -\lfloor (K_\alpha-1)/2 \rfloor, \ldots, -2, -1]$
corresponding to reciprocal vectors, $\vec m \equiv L^{-1} \vec n$.
The box vectors are given by the rows of the lower-diagonal matrix,
$L$, so that the atom coordinates are $\vec r_a = L^T \vec s_a$,
for $s_a \in [0,1)^3$.

  The approximation used is from Essmann, Perera, Berkowitz,
Darden, Lee, and Pedersen, (JCP 103, 1995):
\begin{align}
    \hat S(\vec n) &= \mathcal F[Q](\vec n)
                    / (b_1(\vec n_1) b_2(\vec n_2) b_3(\vec n_3)) \label{e:hS} \\
b_\alpha(n) &= \sum_{j=1}^{r-1} M_r(j) e^{-2\pi i (j-r/2) n/K_\alpha} \\
Q(\vec n) &= \sum_{a=1}^N q_a \prod_{\alpha=1}^3 \left(
    \sum_{t\in\mathbb Z} M_r(\vec n_\alpha - K_\alpha (t+\vec s_{a,\alpha}) + r/2) \right)
.
\end{align}
Here, $M_r$ is an $r^\text{th}$ order Cardinal B-spline defined on $[0,r)$
(implicitly zero elsewhere),
and $\mathcal F[Q]$ denotes the 3D forward DFT (in the FFTW r2c convention)
of the $K_1 \times K_2 \times K_3$ array, $Q$.
This implementation differs from the reference above in using the negative
sign for the frequencies in $S$ and shifting the evaluation points
of $Q(\vec n)$ so as to eliminate the phase of $b$.  This
shift makes it evident that $Q$ is a convolution of the B-spline
smoothing function with the charge density.

  To implement the reciprocal-space part of the Ewald energy, it also
calculates convolutions,
\begin{equation}
E = \frac{1}{2 V} \sum_{\vec n} f(\vec m^2) |\hat S(\vec n)|^2
\end{equation}
(for radially symmetric functions, $f$)
along with their derivatives,
\begin{align}
V \Pi_{\alpha\beta} &= -\sum_\gamma \frac{\partial E}{\partial L_{\gamma\alpha}}
            L_{\gamma\beta} \\
  &= \delta_{\alpha\beta} E
        + \frac{1}{V} \sum_{\vec n}
                            \frac{\partial f(\vec m^2)}{\partial \vec m^2}
            \vec m_\alpha \vec m_\beta |\hat S(\vec n)|^2 \\
\intertext{and}
\frac{\partial E}{\partial \vec r_a} &= \sum_{\vec k}
  \frac{\partial Q(\vec k)}{\partial \vec r_a}
                    \mathcal F^{-1}\left[
                        \frac{f(\vec m^2) \mathcal F[Q](\vec m)}{|B(\vec m)|^2}
                    \right](\vec k)
\end{align}
The volume, $V$, is just the determinant of $L$.
$B$ is the product of $b_1 b_2 b_3$ as in Eq.~\ref{e:hS}.

  For the Ewald sum specifically (in CGS units),
\begin{equation}
f(z) = \exp(-z \pi^2/\eta^2) / \pi z,
\end{equation}
where $\eta$ is the range separation parameter in units of inverse distance.
The user should also add the correction energy,
\begin{equation}
E_\text{corr} = -\frac{\eta}{\sqrt{\pi}} \sum_a q_a^2
,
\end{equation}
so that the result is independent of $\eta$.  Further corrections
are needed if $\sum q_a \ne 0$.

\end{document}
